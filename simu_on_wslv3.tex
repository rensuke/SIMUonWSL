\documentclass[autodetect-engine,dvipdfmx-if-dvi,ja=standard,a4j]{bxjsarticle}
\usepackage {graphicx, xcolor}
	\title{SimutransをWindows Subsystem for Linuxでコンパイルする}
	\author{廉(@osukoke)}
	\date{\today}

\begin{document}
	\maketitle
	\begin{abstract}
		Windows 10より追加された機能、Windows Subsystem for Linux.
		これがアプリとして公開されたことにより、Windows上にLinux環境を構築することが容易になったと考えられる。
		当文書はこの機能を用いてSimutransをコンパイルするまでの手順書を目指して作成している。
	\end{abstract}
	\section{最終目標と実行環境}
		Windows 10の新機能として追加されたWindows Subsystem for Linux(以後WSL)。簡単に言えば、WindowsへLinux環境を構築するアプリである。\par \noindent
		詳細はMicrosoft公式や各種情報サイトを参考にして頂いて、当文書ではWSLの準備からコンパイルまでを紹介する。\par
		今回紹介する方法は、WSLを用いてWindows版Simutransを作成するクロスコンパイルとなる。Linux版などについては対象外としている。\par
		記事内で使用しているパソコンのスペック及び導入するディストリビューションは下記の通り。\par\noindent
		\begin{center}	
			\begin{tabular}{rl}
				WSLディストリビューション & : Ubuntu\\
				機種 & : Lenovo ThinkPad X250\\
				OS & : MicroSoft Windows 10 Pro\\
				CPU & : Intel Core i5-5200U @ 2.20GHz\\
				RAM & : 8GB\\
			\end{tabular}
		\end{center}
	\par
		なお、特記無き場合下記コマンドのバージョン番号は2018年8月10日時点での最新版のため、実行時には適宜最新版に読み替えて実行して頂きたい。\par
	\section{Windows Subsystem for Linuxのセットアップ}
		\subsection{Windows機能のセット}
			初回利用時は、システム設定を変更するために管理者権限を有するアカウントでログインをする。\par\noindent
			スタートメニューを右クリックし「アプリと機能( \underline{F} )」を選択する。\par
			次に、関連設定の「プログラムと機能」をクリックし、コントロールパネルの「プログラムと機能」を呼び出し、プログラムと機能画面左の「Windowsの機能の有効化または無効化」をクリックする。\par\noindent
			「Windows Subsystem for Linux」のチェックボックスをオンにしOKをクリックする。\par
			設定変更が行われ、再起動を求められるため再起動する。
		\subsection{Linuxのインストール}
			Microsoft Storeを開き、検索窓へ「linux」と入力し検索ボタン(虫眼鏡)を押すと複数のディストリビューションが表示される。
			今回はUbuntuを導入するため、Ubuntuをクリックし次の画面で「入手」をクリックする。するとインストールが始まる。\par
			インストールが終了したらスタートメニューからUbuntuを探しクリック(またはストア画面から起動を押す)。
			初回起動時はファイルの展開が行われるためしばらく待つ。\par\noindent
			ファイルの展開が終了するとユーザ名を聞かれるため、Ubuntu用のユーザ名を入力する。このときWindowsとそろえる必要は無いが英数字で登録する。\par\noindent
			ユーザ名を入力しEnterすればパスワードを聞かれる。パスワードは入力しても何も表示されない為注意。\par\noindent
			再度パスワード入力して登録完了。\par
			{\color[rgb]{green}(ユーザ名)@(端末名)}\par
			が表示されればセットアップ完了である。\par
			ディレクトリ(フォルダ)の位置について\par
			~(チルダ)はユーザのホームディレクトリで、Windowsの C:¥Users¥hogeに相当する\par
			ディレクトリを移動するとこの部分が変わる。最上位は/で表される。WindowsでいうCドライブに相当する。\par
			Windowsのディレクトリは、/mnt/c以下にリンクされている。\par
			導入完了後は英語環境となっているため、日本語環境にする場合は下記の手順で行う。(必要に応じ、Japanese Teamサイト参考にパッケージリストの追加)\par
			パッケージ更新(apt update ののち apt upgrade)
			言語パックのインストール
			ロケールの変更>WSLの再起動
			タイムゾーン変更
			マニュアルのインストール
			詳細は、@ITの記事及びUbuntu Japanese Teamのサイトを参照願いたい。なお、Ubuntuバージョンの確認方法は cat /etc/issuieで行える。\par

			[2]パッケージのセットアップ
				Qiitaの記事が詳しいが、Ubuntuで行うにはパッケージの導入に工夫が必要である。
				(1)aptでの導入
					Qiita記事で用いているArch Linuxでは、開発用パッケージをまとめたbase-develパッケージがあるが、Ubuntuでは現状そのようなセットは無いためbase-devel内の各パッケージを指定して導入する。
					Arch Linux公式にリストがあるためそこを参考にaptで導入する。
					なお、zlibやbzipはmingw用に編集して導入する必要があるためこの段階では導入しない。
(表を挿入する)
					また、これとは別にgitなどのパッケージも導入する
(コマンドを挿入)
				(2)zip系の導入
					/usr/local/srcへクロスコンパイル用に設定したzlib、bzip2を導入する。
					共通作業:srcディレクトリへ移動する
						cd /usr/local/src
					1.zlib
						zlib公式サイトより最新版をダウンロードする。
							sudo wget http://zlib.net/zlib-1.2.11.tar.gz
						解凍し、ディレクトリを移動する
							sudo tar xvf zlib-1.2.11.tar.gz
							cd zlib-1.2.11
						環境変数をクロスコンパイル用に設定してmakeを実行する。
							PREFIXDIR=/usr/x86_64-w64-mingw32
							sudo make -f win32/Makefile.gcc BINARY_PATH=$PREFIXDIR/bin INCLUDE_PATH=$PREFIXDIR/include LIBRARY_PATH=$PREFIXDIR/lib SHARED_MODE=1 PREFIX=x86_64-w64-mingw32- install
					2.bzip
						現在、公式サイトは期限切れにより不審サイト化しておりパッケージをダウンロードできない。
						そのため、Ubuntuのパッケージ検索サイトよりダウンロードする。
							sudo wget http://archive.ubuntu.com/ubuntu/pool/main/b/bzip2/bzip2_1.0.6.orig.tar.bz2
						zlibと同様に解凍し、ディレクトリを移動する。コマンドは先述のzlibと同様のため省略。
						ただし、解凍後のディレクトリ名が異なるため、lsコマンドでディレクトリ一覧を取得し確認後に移動する。
						移動後、幾つかのファイル内容をクロスコンパイル用に編集する。
						bzip2.cは、sys\stat.hの\(逆スラッシュ)を/(スラッシュ)へ変える。
						Makefileは、変数設定をMingw系に変更する。
						これらのコマンドは下記の通りである。
							sudo sed -i 's|sys\\stat.h|sys/stat.h|g' bzip2.c
							sudo sed -i 's|CC=gcc|CC=x86_64-w64-mingw32-gcc|g' Makefile
							sudo sed -i 's|AR=ar|AR=x86_64-w64-mingw32-ar|g' Makefile
							sudo sed -i 's|RANLIB=ranlib|RANLIB=x86_64-w64-mingw32-ranlib|g' Makefile
							sudo sed -i 's|PREFIX=/usr/local|PREFIX=/usr/x86_64-w64-mingw32|g' Makefile



[参照記事]
 Tech TIPS:WSLのUbuntu環境を日本語化する(@IT)
	http://www.atmarkit.co.jp/ait/articles/1806/28/news043.html
Tech TIPS:Windows 10でLinuxプログラムを利用可能にするWSL(Windows Subsystem for Linux)をインストールする(バージョン1803対応版)(@IT)
	http://www.atmarkit.co.jp/ait/articles/1608/08/news039.html
Ubuntuの日本語環境(Ubuntu Japanese Team)
	http://www.ubuntulinux.jp/japanese
Simutransをビルドしてみる(Qiita)
	https://qiita.com/Aruneko/items/373ed7d135b6f686dbd9
Arch Linux - base-devel(x86_64) - Group Details
	https://www.archlinux.org/groups/x86_64/base-devel/
絶対領域(AbsoluteArea)の徒然-Windows 64bit 用のffmpegをビルド-(絶対領域氏)
	http://absolutearea.blogspot.jp/2012/09/windows-64bit-ffmpeg.html
なんとな~くしあわせ?の日記-Debian Wheezy上でWindows-x64向けバイナリを作成する-(Hiroguki-Nagata氏)
	http://nantonaku-shiawase.hatenablog.com/entry/2013/04/25/003059

\end{document}

